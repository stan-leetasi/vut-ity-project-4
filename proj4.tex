%Autor: Stanislav Letaši xletas00
\documentclass[a4paper, 11pt]{article}
\usepackage[czech,slovak]{babel}
\usepackage[IL2]{fontenc}
\usepackage[a4paper, total={170mm, 240mm}, top=3cm, left=2cm]{geometry}
\usepackage[utf8]{inputenc}
\usepackage{url}
\usepackage[hidelinks,breaklinks]{hyperref}
\begin{document}

\begin{titlepage}
   \begin{center}
    \Huge
    
    {\textsc{Vysoké učení technické v~Brně}}\\
    {\textsc{\huge Fakulta informačních technologií}}\\
    \vspace{\stretch{0.382}}
    
    \LARGE
    Typografie a publikování\,--\,4. projekt\\
    \Huge Kybernetická bezpečnosť
       
    \vspace{\stretch{0.618}}

    \Large \today \hspace{\stretch{1}} Stanislav Letaši
            
   \end{center}
\end{titlepage}

\section{Úvod}
Bezpečnosť na internete je téma, ktorá v~dnešnom digitálnom svete nadobúda každým dňom stále väčšiu podstatu. V~dnešnej dobe je internet integrovaný do každodenného života čoraz viac, čo znamená, že ľudia sa musia učiť chrániť si svoje osobné údaje pred neoprávneným prístupom. 

\section{Ako sa ochrániť}
Výskumníci z~oblasti kybernetickej bezpečnosti zdôrazňujú, že 
je dôležité používať aktualizovaný antivírusový software a 
firewall pre ochranu počítača pred útokmi z~internetu \cite{johnson2015cybersecurity}. Taktiež spomínajú, že chrániť 
si svoje citlivé informácie pomocou silných hesiel a správneho 
šifrovania je v~dnešnej dobe \underline{nevyhnutné}. 
Slabé heslá môžu byť totiž častokrát zneužité hackermi, ktorí 
dokážu pomocou slovníkového útoku získať prístup na osobný účet napadnutej osoby, alebo do systému cieľovej organizácie~\cite{hodes2015moderni}. {\it Journal of Cybersecurity} zverejnil 
článok, v~ktorom sa autori zaoberajú problematikou vytvárania 
bezpečných hesiel na internete \cite{10.1093/cybsec/tyab012}. 
Štúdia Oxfordskej univerzity, ktorú článok spomína zistila, že 
väčšina ľudia je v~dnešnej dobe oveľa viac obozretná, a dbajú 
na vytváranie si silnejších hesiel a nepoužívanie rovnakého 
hesla na viacerých stránkach. 

\section{Wiper malware}
V~roku 2022 boli pre digitálny svet najväčšou hrozbou takzvané 
\uv{wiper attacks}. Ako je spomenuté v~článku z~{\it Infosecurity Magazine} \cite{poireault-infosecmagazine-q1-2023}, \uv{wipers} 
sú formou malware, ktorý nenávratne prepisuje dáta. Tieto útoky 
boli využité niektorými hackerskými skupinami pre finančný zisk 
tým, že sa vyhrážali postupným vymazaním systému, pokiaľ im 
napadnutá strana nevyplatí žiadané výkupné.

\section{Nové technológie}
Ziska Fields v~svojej knihe \cite{fields2018handbook} spomína, že v~dnešnej dobe rýchleho vývoja technológií, ktoré sú súčasťou \uv{štvrtej priemyselnej revolúcie} je kybernetická bezpečnosť stále dôležitejšia. Taktiež zdôrazňuje potrebu rozvoja nových zabezpečovacích opatrení a technológií, ktoré budú schopné ochrániť všetky aspekty kybernetickej bezpečnosti.

Za posledné roky sa kybernetické útoky stávajú viac sofistikované a útočníci začínajú využívať stále novšie technológie a metódy. Preto je dôležité, aby organizácie neustále aktualizovali svoje bezpečnostné opatrenia a sledovali aktuálne trendy v~oblasti kybernetickej bezpečnosti. Podľa {\it Forbes} článku {\it Top Cybersecurity Predictions 2023} \cite{sayegh2022top} bude rozvoj Secure access service edge a Zero Trust Adoption najväčšími pokrokmi tohto roku v~oblasti kybernetickej bezpečnosti.

\section{SASE a Zero Trust}
\textbf{Secure access service edge}\,-\,SASE, je nové riešenie pre prepojenie sietí a ochranu pred bezpečnostnými hrozbami \cite{chen2023overview}. SASE poskytuje užívateľom bezpečný priamy prístup ku cloudu, bez nutnosti pripájania sa prostredníctvom MPLS alebo VPN, a ponúka užívateľom bezpečnú bránu pre webové služby, izoláciu vzdialeného prehliadača a taktiež prevenciu prienikov.

\textbf{Zero trust} je prístup ku kybernetickej bezpečnosti, ktorý vychádza z~myšlienky, že žiadne pripojenie k~firemným sieťam a systémom by nemalo byť považované za dôveryhodné \cite{pratt2022history}. Model zero trust vyžaduje overenie užívateľov, zariadení a systémov pred pripojením a opakované overovanie pri prístupu k~sieťam, systémom a dátam. Podľa nedávnych správ sa stále viac organizácií rozhoduje pre prístup zero trust, a percento organizácií prechádzajúcich k~tomuto prístupu bude naďalej len rásť. 

\section{Stále nebezpečenstvo Phishing útokov}
Aj napriek nespočetným technológiám na ochranu bezpečnosti užívateľov a spoločností je jeden z najčastejších spôsobov ako útočníci získajú prístup do systému phishing. Phishing útok je forma sociálneho inžinierstva, v~ktorej sa útočník vydáva za dôveryhodnú osobu a snaží sa od obete získať citlivé informácie alebo prístup do zabezpečeného systému \cite{Dohnal2022thesis}. V~snahe zabrániť týmto útokom bolo vyvinutých veľa ochranných systémov, najnovšie systémy používajúce umelú inteligenciu na rozpoznanie varovných signálov nachádzajúcich sa vo phishingových správach \cite{sennovate:2023}.

\section{Záver}
S~vývojom nových technológií na ochranu pred kybernetickými útokmi bohužiaľ prichádzajú aj nové spôsoby ako tieto ochrany obísť. Neopatrnosť len jedného človeka môže viesť napríklad k~napadnutiu vnútorného systému spoločnosti, v~ktorej pracuje. Preto je stále relevantné zostať obozretný a riadiť sa základnými pravidlami internetovej bezpečnosti, ako používanie silných hesiel, vyhýbanie sa podozrivým stránkam a sťahovanie súborov z~neoverených zdrojov. 

\bibliographystyle{czechiso}
\bibliography{citations.bib} 

\end{document}
